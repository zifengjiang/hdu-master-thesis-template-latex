\documentclass[oneside, nocpsupervisor, blind]{theme/hduthesis} % 如果你的论文不满80页,还是单面印刷吧

\newcommand{\mycite}[1]{\scalebox{1.3}[1.3]{\raisebox{-0.65ex}{\cite{#1}}}}

\newcommand*\circled[1]{\tikz[baseline=(char.base)]{\node[shape=circle,draw,inner sep=0.15ex] (char) {#1};}}


\usepackage{makecell}
\usepackage{diagbox}
\usepackage{pdfpages}
\title{基于xxxxxxxx方法研究}
\englishtitle{xxxxxxx} % 封面和题名页使用


\author{xx}% 申请人姓名 封面使用
\englishauthor{xxxxxx}

\classification{TP311.1}    % 封面头使用
\serialnumber{10335}        % 封面头使用
\secretlevel{无}            % 封面头使用
\studentnumber{123456}    % 封面头使用

\supervisor{xxxx}
\englishsupervisor{xxxxx}
\englishspvtitle{Prof.}     % 职称 封面使用
\spvtitle{教授}             % 职称 封面使用
   
% \cpsupervisor{合作导师}      % 合作导师,如果没有合作导师,就在此文件第 4 行\documentclass选项栏中加上"nocpsupervisor"。
% \cspvtitle{副教授}           % 合作导师职称

% 从机械工程学院改来,保留设定变量命名
\major{xxxx}
\research{xxxxx}
\institute{xxxxxx}    % 所在学位栏 填 软件学院

\submitdate{2025年3月}   % 论文提交日期 栏

% 答辩日期
\defencedate{2025年3月}
\eendefencedate{March, 2025}  % 因为endefencedate 命名被占用

% 论文前置部分变量填写完毕 开始全书排版
\begin{document}

% 封面、中文题名页、英文题名页、独创声明和版权使用书 无页码
\maketitle


% 摘要部分
\abstractmatter
%\includepdf{论文原创声明和使用授权说明.pdf}
\begin{cabstract}

xxxx

\end{cabstract}

\ckeywords{xxxx,xxxx,xxxx,xxxx,xxxx}
\begin{eabstract}

xxxxxxxxxxxxxxxx

\end{eabstract}

\ekeywords{xxxx, xxxx, xxxx, xxxx}

% 目录和术语表
\frontmatter
\tableofcontents    % 正文目录
%\listoffigures    % 图目录
%\listoftables     % 表目录
% 术语及缩略词表(需要则开)
%\chapter{符号与函数说明表}

\begin{table*}[htb]
    \centering
    \small
    \caption{符号与函数说明表}
    \begin{tabular}{ll}
        \toprule
        符号与函数      & 说明         \\
        \midrule
        MoCo           & 在2020年发表的一种对比学习方法,全称momentum contrast learning。 \\
        InfoNCE        & \makecell[l]{名为信息噪声对比估计(information noise contrastive estimation),是一种常用于计算 \\ 对比损失的方法。} \\
        q              & 表示查询图像的特征向量。\\
        $k_+$          & 表示正样本的特征向量。 \\
        $k_i$          & 表示第$i$个负样本的特征向量。 \\
        $\mathbf{K}$   & 表示 \\
        M              &  \\
        N              &  \\
        m              &  \\
        n              &  \\
        d              &  \\
        $L_coo$        &  \\
        $L_con$        &  \\
        warmup         &  \\
        threadIdx      &  \\
        blockDim       &  \\
        blockIdx       &  \\
        tid            &  \\
        Coin           &  \\
        IsEqual        &  \\
        Uniform        &  \\
        Shuffle        &  \\
        Transpose      &  \\
        Arange         &  \\
        Conv2d         &  \\
        RInt           &  \\
        \bottomrule
    \end{tabular}
\end{table*}

% 正文排版开始 建议一章一编辑 (好像无法嵌套 include) 
\mainmatter
\chapter{绪论}
\section{研究背景与意义}



\chapter{相关理论}


 
\chapter{研究一}


\chapter{研究二}


\chapter{研究三}


\chapter{总结和展望}


% 结尾部分排版
\backmatter
\makeatletter
\if@blind
    % 盲审模式,不包含致谢
\else
    % !TEX root = ../main.tex
\chapter{致\HDUspace{}谢}



\vspace{1cm}
\hfill
\begin{minipage}{14em}
    \makeatletter
    \begin{flushright}
        \HDU@author\\
        \HDU@defenceyear 年 \HDU@defencemonth 月 \HDU@defenceday 日
    \end{flushright}
    \makeatother
\end{minipage}
 % 非盲审模式才包含致谢
\fi
\makeatother
% 引用参考文献数据库
\bibliographystyle{theme/gbt7714-unsrt-srcname}
\bibliography{references/ref.bib}

% 附录部分
\appendix
%\chapter{基于GPU加速的随机数据增强的预处理模块}

% 作者简历
\makeatletter
\if@blind
	
\cleardoublepage
\pdfbookmark{附录}{anchor_app}

\vspace*{-3.5mm}
{
\centering
	\heiti\zihao{3}\bfseries·附录\par
}

\vspace{6.2mm}

{\centering\heiti\zihao{3}\bfseries 攻读硕士学位期间取得的研究成果\par}

\vspace{8mm}


{
\setlength{\parindent}{3em}
\linespread{2.0}
\indent 一、已发表(包括已录用待发表)的论文\par
}

\vspace{2mm}
{
\centering\linespread{1.2}
  \begin{tabular}{
  | >{\centering\arraybackslash}m{0.49cm}<{\justifying} 
  | >{\centering\arraybackslash}m{5.94cm}<{\centering}  
  | >{\centering\arraybackslash}m{2.05cm}<{\justifying} 
  | >{\centering\arraybackslash}m{1.64cm}<{\justifying}
  | >{\centering\arraybackslash}m{2.05cm}<{\justifying} 
  | >{\centering\arraybackslash}m{1.64cm}<{\justifying} |}
    \hline
    % 表头:宋体、小四、加粗,两端对齐(这里使用居中效果),行高0.33cm
    {\songti\zihao{-4}\bfseries 序号\strut} \rule{0pt}{0.33cm} &
    {\songti\zihao{-4}\bfseries 发表或投稿刊物/会议名称\strut} \rule{0pt}{0.33cm} &
    {\songti\zihao{-4}\bfseries 作者(仅注明第几作者)\strut} \rule{0pt}{0.33cm} &
    {\songti\zihao{-4}\bfseries 发表年份\strut} \rule{0pt}{0.33cm} &
    {\songti\zihao{-4}\bfseries 与学位论文哪一部分(章、节)相关\strut} \rule{0pt}{0.33cm} &
    {\songti\zihao{-4}\bfseries 被索引收录情况\strut} \rule{0pt}{0.33cm} \\
    \hline
    % 以下行设置为1.55cm行高
    {\songti\zihao{-4}\bfseries \strut} \rule{0pt}{1.55cm} &
    {\songti\zihao{-4}\bfseries \strut} \rule{0pt}{1.55cm} &
    {\songti\zihao{-4}\bfseries \strut} \rule{0pt}{1.55cm} &
    {\songti\zihao{-4}\bfseries \strut} \rule{0pt}{1.55cm} &
    {\songti\zihao{-4}\bfseries \strut} \rule{0pt}{1.55cm} &
    {\songti\zihao{-4}\bfseries \strut} \rule{0pt}{1.55cm} \\
    \hline
    
    {\songti\zihao{-4}\bfseries \strut} \rule{0pt}{1.55cm} &
    {\songti\zihao{-4}\bfseries \strut} \rule{0pt}{1.55cm} &
    {\songti\zihao{-4}\bfseries \strut} \rule{0pt}{1.55cm} &
    {\songti\zihao{-4}\bfseries \strut} \rule{0pt}{1.55cm} &
    {\songti\zihao{-4}\bfseries \strut} \rule{0pt}{1.55cm} &
    {\songti\zihao{-4}\bfseries \strut} \rule{0pt}{1.55cm} \\
    \hline

	{\songti\zihao{-4}\bfseries \strut} \rule{0pt}{1.55cm} &
    {\songti\zihao{-4}\bfseries \strut} \rule{0pt}{1.55cm} &
    {\songti\zihao{-4}\bfseries \strut} \rule{0pt}{1.55cm} &
    {\songti\zihao{-4}\bfseries \strut} \rule{0pt}{1.55cm} &
    {\songti\zihao{-4}\bfseries \strut} \rule{0pt}{1.55cm} &
    {\songti\zihao{-4}\bfseries \strut} \rule{0pt}{1.55cm} \\
    \hline
	
	{\songti\zihao{-4}\bfseries \strut} \rule{0pt}{1.55cm} &
    {\songti\zihao{-4}\bfseries \strut} \rule{0pt}{1.55cm} &
    {\songti\zihao{-4}\bfseries \strut} \rule{0pt}{1.55cm} &
    {\songti\zihao{-4}\bfseries \strut} \rule{0pt}{1.55cm} &
    {\songti\zihao{-4}\bfseries \strut} \rule{0pt}{1.55cm} &
    {\songti\zihao{-4}\bfseries \strut} \rule{0pt}{1.55cm} \\
    \hline

	{\songti\zihao{-4}\bfseries \strut} \rule{0pt}{1.55cm} &
    {\songti\zihao{-4}\bfseries \strut} \rule{0pt}{1.55cm} &
    {\songti\zihao{-4}\bfseries \strut} \rule{0pt}{1.55cm} &
    {\songti\zihao{-4}\bfseries \strut} \rule{0pt}{1.55cm} &
    {\songti\zihao{-4}\bfseries \strut} \rule{0pt}{1.55cm} &
    {\songti\zihao{-4}\bfseries \strut} \rule{0pt}{1.55cm} \\
    \hline

  \end{tabular}
}
	  
  \vspace{5mm}
  

{
\setlength{\parindent}{3em}
\linespread{2.0}
\indent 二、与学位论文内容相关的其它成果(包括专利、著作、获奖项目等)\par
\indent 专利:已授权一项发明专利,公开一项发明专利,第一/一发明人,2030年11月/2030年12月\par
\indent 竞赛:获xxxxx竞赛全国一等奖,第一获奖人,2099年1月。\par
}

\else
	% !TEX root = ../main.tex
\chapter{作者在读期间发表的学术论文及参加的科研项目}




\noindent{\textbf{一、研究生期间申请的专利:}}

\begin{enumerate}[{[1]}]
	\item{xxxxxxxxxxxxxxxxxxxxxxxx, 申请公布号:xxxxxxxxxxxx, 申请公布日:2029-12-05. (导师一作,本人二作)}

\end{enumerate}


\fi
\makeatother
\end{document}
